\documentclass[a4paper]{extarticle}
\usepackage[utf8]{inputenc}
\usepackage[T1]{fontenc}
\usepackage[italian]{babel}
\selectlanguage{italian}
\usepackage[table]{xcolor}
\usepackage{xcolor}
\usepackage{circuitikz}
\usetikzlibrary{positioning, circuits.logic.US}
\usetikzlibrary {shapes.gates.logic.US, shapes.gates.logic.IEC, calc}
\tikzset {branch/.style={fill, shape = circle, minimum size = 3pt, inner sep = 0pt}}
\usetikzlibrary{matrix,calc}
\usepackage{multirow}
\usepackage{float}
\usepackage{geometry}
\usepackage{tabularx}
\usepackage{pgf-pie}
\usepackage{tikz}
\usepackage{amsmath}
\usepackage{amssymb}
\usepackage{color, soul}
\usepackage{fancyhdr}
\usepackage{graphicx}
\usepackage{subfig}
\graphicspath{ {./img/} }
\newtheorem{theorem}{Teorema}[section]
\newtheorem{corollary}{Corollario}[theorem]
\newtheorem{lemma}[theorem]{Lemma}

% Specifiche
\geometry{
 a4paper,
 top=20mm,
 left=30mm,
 right=30mm,
 bottom=30mm
}

\pagestyle{fancy}
\fancyhf{}
\fancyhead[LO]{\nouppercase{\leftmark}}
\fancyfoot[CE, CO]{\thepage}
\addtolength{\headheight}{1em}
\addtolength{\footskip}{-0.5em}

\newcommand{\quotes}[1]{``#1''}
\renewcommand\tabularxcolumn[1]{>{\vspace{\fill}}m{#1}<{\vspace{\fill}}}
\renewcommand\arraystretch{}
\newcolumntype{P}{>{\centering\arraybackslash}X}

\newcommand{\horrule}[1]{\noindent\rule{\linewidth}{#1}}

\title{
\normalfont \normalsize
\textsc{Università degli Studi di Trieste - Dipartimento di Ingegneria e Architettura} \\[25pt]
\horrule{0.5pt} \\ [0.4cm] % Thin top horizontal rule
\huge Progetto - Fondamenti d'Informatica  \\ % The assignment title
\horrule{2pt} \\ [0.4cm]% Thick bottom horizontal rule
}
\author{Enrico Piccin - IN0501089}
\date{Anno Accademico 2021/2022}

%isolated term
%#1 - Optional. Space between node and grouping line. Default=0
%#2 - node
%#3 - filling color
\newcommand{\implicantsol}[3][0]{
    \draw[rounded corners=3pt, fill=#3, opacity=0.3] ($(#2.north west)+(135:#1)$) rectangle ($(#2.south east)+(-45:#1)$);
    }


%internal group
%#1 - Optional. Space between node and grouping line. Default=0
%#2 - top left node
%#3 - bottom right node
%#4 - filling color
\newcommand{\implicant}[4][0]{
    \draw[rounded corners=3pt, fill=#4, opacity=0.3] ($(#2.north west)+(135:#1)$) rectangle ($(#3.south east)+(-45:#1)$);
    }

%group lateral borders
%#1 - Optional. Space between node and grouping line. Default=0
%#2 - top left node
%#3 - bottom right node
%#4 - filling color
\newcommand{\implicantcostats}[4][0]{
    \draw[rounded corners=3pt, fill=#4, opacity=0.3] ($(rf.east |- #2.north)+(90:#1)$)-| ($(#2.east)+(0:#1)$) |- ($(rf.east |- #3.south)+(-90:#1)$);
    \draw[rounded corners=3pt, fill=#4, opacity=0.3] ($(cf.west |- #2.north)+(90:#1)$) -| ($(#3.west)+(180:#1)$) |- ($(cf.west |- #3.south)+(-90:#1)$);
}

%group top-bottom borders
%#1 - Optional. Space between node and grouping line. Default=0
%#2 - top left node
%#3 - bottom right node
%#4 - filling color
\newcommand{\implicantdaltbaix}[4][0]{
    \draw[rounded corners=3pt, fill=#4, opacity=0.3] ($(cf.south -| #2.west)+(180:#1)$) |- ($(#2.south)+(-90:#1)$) -| ($(cf.south -| #3.east)+(0:#1)$);
    \draw[rounded corners=3pt, fill=#4, opacity=0.3] ($(rf.north -| #2.west)+(180:#1)$) |- ($(#3.north)+(90:#1)$) -| ($(rf.north -| #3.east)+(0:#1)$);
}

%group corners
%#1 - Optional. Space between node and grouping line. Default=0
%#2 - filling color
\newcommand{\implicantcantons}[2][0]{
    \draw[rounded corners=3pt, opacity=.3] ($(rf.east |- 0.south)+(-90:#1)$) -| ($(0.east |- cf.south)+(0:#1)$);
    \draw[rounded corners=3pt, opacity=.3] ($(rf.east |- 8.north)+(90:#1)$) -| ($(8.east |- rf.north)+(0:#1)$);
    \draw[rounded corners=3pt, opacity=.3] ($(cf.west |- 2.south)+(-90:#1)$) -| ($(2.west |- cf.south)+(180:#1)$);
    \draw[rounded corners=3pt, opacity=.3] ($(cf.west |- 10.north)+(90:#1)$) -| ($(10.west |- rf.north)+(180:#1)$);
    \fill[rounded corners=3pt, fill=#2, opacity=.3] ($(rf.east |- 0.south)+(-90:#1)$) -|  ($(0.east |- cf.south)+(0:#1)$) [sharp corners] ($(rf.east |- 0.south)+(-90:#1)$) |-  ($(0.east |- cf.south)+(0:#1)$) ;
    \fill[rounded corners=3pt, fill=#2, opacity=.3] ($(rf.east |- 8.north)+(90:#1)$) -| ($(8.east |- rf.north)+(0:#1)$) [sharp corners] ($(rf.east |- 8.north)+(90:#1)$) |- ($(8.east |- rf.north)+(0:#1)$) ;
    \fill[rounded corners=3pt, fill=#2, opacity=.3] ($(cf.west |- 2.south)+(-90:#1)$) -| ($(2.west |- cf.south)+(180:#1)$) [sharp corners]($(cf.west |- 2.south)+(-90:#1)$) |- ($(2.west |- cf.south)+(180:#1)$) ;
    \fill[rounded corners=3pt, fill=#2, opacity=.3] ($(cf.west |- 10.north)+(90:#1)$) -| ($(10.west |- rf.north)+(180:#1)$) [sharp corners] ($(cf.west |- 10.north)+(90:#1)$) |- ($(10.west |- rf.north)+(180:#1)$) ;
}

%Empty Karnaugh map 4x4
\newenvironment{Karnaugh}%
{
\begin{tikzpicture}[baseline=(current bounding box.north),scale=0.8]
\draw (0,0) grid (4,4);
\draw (0,4) -- node [pos=0.7,above right,anchor=south west] {$xy$} node [pos=0.7,below left,anchor=north east] {$zw$} ++(135:1);
%
\matrix (mapa) [matrix of nodes,
        column sep={0.8cm,between origins},
        row sep={0.8cm,between origins},
        every node/.style={minimum size=0.3mm},
        anchor=8.center,
        ampersand replacement=\&] at (0.5,0.5)
{
                       \& |(c00)| $00$         \& |(c01)| $01$         \& |(c11)| $11$         \& |(c10)| $10$         \& |(cf)| \phantom{00} \\
|(r00)| $00$             \& |(0)|  \phantom{0} \& |(1)|  \phantom{0} \& |(3)|  \phantom{0} \& |(2)|  \phantom{0} \&                     \\
|(r01)| $01$             \& |(4)|  \phantom{0} \& |(5)|  \phantom{0} \& |(7)|  \phantom{0} \& |(6)|  \phantom{0} \&                     \\
|(r11)| $11$             \& |(12)| \phantom{0} \& |(13)| \phantom{0} \& |(15)| \phantom{0} \& |(14)| \phantom{0} \&                     \\
|(r10)| $10$             \& |(8)|  \phantom{0} \& |(9)|  \phantom{0} \& |(11)| \phantom{0} \& |(10)| \phantom{0} \&                     \\
|(rf) | \phantom{00}   \&                    \&                    \&                    \&                    \&                     \\
};
}%
{
\end{tikzpicture}
}

%Empty Karnaugh map 2x4
\newenvironment{Karnaughvuit}%
{
\begin{tikzpicture}[baseline=(current bounding box.north),scale=0.8]
\draw (0,0) grid (4,2);
\draw (0,2) -- node [pos=0.7,above right,anchor=south west] {$xy$} node [pos=0.7,below left,anchor=north east] {$z$} ++(135:1);
%
\matrix (mapa) [matrix of nodes,
        column sep={0.8cm,between origins},
        row sep={0.8cm,between origins},
        every node/.style={minimum size=0.3mm},
        anchor=4.center,
        ampersand replacement=\&] at (0.5,0.5)
{
                      \& |(c00)| $00$         \& |(c01)| $01$         \& |(c11)| $11$         \& |(c10)| $10$         \& |(cf)| \phantom{00} \\
|(r00)| $0$             \& |(0)|  \phantom{0} \& |(1)|  \phantom{0} \& |(3)|  \phantom{0} \& |(2)|  \phantom{0} \&                     \\
|(r01)| $1$             \& |(4)|  \phantom{0} \& |(5)|  \phantom{0} \& |(7)|  \phantom{0} \& |(6)|  \phantom{0} \&                     \\
|(rf) | \phantom{00}  \&                    \&                    \&                    \&                    \&                     \\
};
}%
{
\end{tikzpicture}
}

%Empty Karnaugh map 2x2
\newenvironment{Karnaughquatre}%
{
\begin{tikzpicture}[baseline=(current bounding box.north),scale=0.8]
\draw (0,0) grid (2,2);
\draw (0,2) -- node [pos=0.7,above right,anchor=south west] {$x$} node [pos=0.7,below left,anchor=north east] {$y$} ++(135:1);
%
\matrix (mapa) [matrix of nodes,
        column sep={0.8cm,between origins},
        row sep={0.8cm,between origins},
        every node/.style={minimum size=0.3mm},
        anchor=2.center,
        ampersand replacement=\&] at (0.5,0.5)
{
          \& |(c00)| $0$          \& |(c01)| $1$  \\
|(r00)| $0$ \& |(0)|  \phantom{0} \& |(1)|  \phantom{0} \\
|(r01)| $1$ \& |(2)|  \phantom{0} \& |(3)|  \phantom{0} \\
};
}%
{
\end{tikzpicture}
}

%Defines 8 or 16 values (0,1,X)
\newcommand{\contingut}[1]{%
\foreach \x [count=\xi from 0]  in {#1}
     \path (\xi) node {$\x$};
}

%Places 1 in listed positions
\newcommand{\minterms}[1]{%
    \foreach \x in {#1}
        \path (\x) node {1};
}

%Places 0 in listed positions
\newcommand{\maxterms}[1]{%
    \foreach \x in {#1}
        \path (\x) node {0};
}

%Places X in listed positions
\newcommand{\indeterminats}[1]{%
    \foreach \x in {#1}
        \path (\x) node {X};
}

\begin{document}

\vspace{-10mm}
\maketitle

\tableofcontents
\newpage

\section{Individuazione della funzione Booleana}
A partire dalla matricola IN0501089, si procede elidenndo il prefisso IN ed individuando il numero di matricola associato: \textbf{0501089}.\\
Dividendo tale numero per \(2^{2^4} = 65536\) si perviene al risultato seguente:
\[\frac{501089}{65536} = 7 + \textbf{42337}\]
Avendo ricavato il resto $42337$, si procede a codificarlo in binario, impiegando $16$ bit, tramite succcessive divisioni del numero ottenuto per $2$, come illustrato di seguito:

\begin{table}[H]
  \centering
  \begin{align*}
    42337 = 21168 \cdot 2 + \boxed{1}\\
    21164 = 10584 \cdot 2 + \boxed{0}\\
    10584 = 5298  \cdot 2 + \boxed{0}\\
     5298 = 2646  \cdot 2 + \boxed{0}\\
     2646 = 1323  \cdot 2 + \boxed{0}\\
     1323 = 661   \cdot 2 + \boxed{1}\\
      661 = 330   \cdot 2 + \boxed{1}\\
      330 = 165   \cdot 2 + \boxed{0}\\
      165 = 82    \cdot 2 + \boxed{1}\\
       82 = 41    \cdot 2 + \boxed{0}\\
       41 = 20    \cdot 2 + \boxed{1}\\
       20 = 10    \cdot 2 + \boxed{0}\\
       10 = 5     \cdot 2 + \boxed{0}\\
        5 = 2     \cdot 2 + \boxed{1}\\
        2 = 1     \cdot 2 + \boxed{0}\\
        1 = 0     \cdot 2 + \boxed{1}
  \end{align*}
  \begin{tabular}{c}
    $42337_{10} = \boldsymbol{1010010101100001_2}$\\
  \end{tabular}
  \caption{Rappresentazione del resto $42337$ in binario}
  \label{tab:rappresentazione_resto_binario}
\end{table}

\newpage
\noindent
Pertanto, la funzione Booleana a \(4\) associata corrisponde alla stringa binaria di cui sopra, da cui si evincono i seguenti termini minimi (\emph{minterm}) e massimi (\emph{maxterm}):

\noindent
\begin{table}[H]
  \setlength{\tabcolsep}{5.6pt}
  \hspace{-1em}
  \begin{tabularx}{\textwidth}{ll}
    {
        \noindent
        \begin{tabular}{c|c||cccc||c|c}
          $ $ & $ $ & $x$ & $y$ & $z$ & $w$ & $\boldsymbol{f}$\\
          \hline
          $m_0$ & $\overline{x}\overline{y}\overline{z}\overline{w}$ & $0$ & $0$ & $0$ & $0$ & $\boldsymbol{1}$ & $\mu_0$\\
          $m_1$ & $\overline{x}\overline{y}\overline{z}w$ & $0$ & $0$ & $0$ & $1$ & $\boldsymbol{0}$ & $\mu_1$\\
          $m_2$ & $\overline{x}\overline{y}z\overline{w}$ & $0$ & $0$ & $1$ & $0$ & $\boldsymbol{1}$ & $\mu_2$\\
          $m_3$ & $\overline{x}\overline{y}zw$ & $0$ & $0$ & $1$ & $1$ & $\boldsymbol{0}$ & $\mu_3$\\
          $m_4$ & $\overline{x}y\overline{z}\overline{w}$ & $0$ & $1$ & $0$ & $0$ & $\boldsymbol{0}$ & $\mu_4$\\
          $m_5$ & $\overline{x}y\overline{z}w$ & $0$ & $1$ & $0$ & $1$ & $\boldsymbol{1}$ & $\mu_5$\\
          $m_6$ & $\overline{x}yz\overline{w}$ & $0$ & $1$ & $1$ & $0$ & $\boldsymbol{0}$ & $\mu_6$\\
          $m_7$ & $\overline{x}yzw$ & $0$ & $1$ & $1$ & $1$ & $\boldsymbol{1}$ & $\mu_7$\\
          $m_8$ & $x\overline{y}\overline{z}\overline{w}$ & $1$ & $0$ & $0$ & $0$ & $\boldsymbol{0}$ & $\mu_8$\\
          $m_9$ & $x\overline{y}\overline{z}w$ & $1$ & $0$ & $0$ & $1$ & $\boldsymbol{1}$ & $\mu_9$\\
          $m_{10}$ & $x\overline{y}z\overline{w}$ & $1$ & $0$ & $1$ & $0$ & $\boldsymbol{1}$ & $\mu_{10}$\\
          $m_{11}$ & $x\overline{y}zw$ & $1$ & $0$ & $1$ & $1$ & $\boldsymbol{0}$ & $\mu_{11}$\\
          $m_{12}$ & $xy\overline{z}\overline{w}$ & $1$ & $1$ & $0$ & $0$ & $\boldsymbol{0}$ & $\mu_{12}$\\
          $m_{13}$ & $xy\overline{z}w$ & $1$ & $1$ & $0$ & $1$ & $\boldsymbol{0}$ & $\mu_{13}$\\
          $m_{14}$ & $xyz\overline{w}$ & $1$ & $1$ & $1$ & $0$ & $\boldsymbol{0}$ & $\mu_{14}$\\
          $m_{15}$ & $xyzw$ & $1$ & $1$ & $1$ & $1$ & $\boldsymbol{1}$ & $\mu_{15}$\\
        \end{tabular}
    }
    &
    {
        \noindent
        \begin{tabular}{c|c||cccc||c|c}
          $ $ & $ $ & $x$ & $y$ & $z$ & $w$ & $\boldsymbol{f}$\\
          \hline
          $M_0$ & $x + y + z + w$ & $0$ & $0$ & $0$ & $0$ & $\boldsymbol{1}$ & $\mu_0$\\
          $M_1$ & $x + y + z + \overline{w}$ & $0$ & $0$ & $0$ & $1$ & $\boldsymbol{0}$ & $\mu_1$\\
          $M_2$ & $x + y + \overline{z} + w$ & $0$ & $0$ & $1$ & $0$ & $\boldsymbol{1}$ & $\mu_2$\\
          $M_3$ & $x + y + \overline{z} + \overline{w}$ & $0$ & $0$ & $1$ & $1$ & $\boldsymbol{0}$ & $\mu_3$\\
          $M_4$ & $x + \overline{y} + z + w$ & $0$ & $1$ & $0$ & $0$ & $\boldsymbol{0}$ & $\mu_4$\\
          $M_5$ & $x + \overline{y} + z + \overline{w}$ & $0$ & $1$ & $0$ & $1$ & $\boldsymbol{1}$ & $\mu_5$\\
          $M_6$ & $x + \overline{y} + \overline{z} + w$ & $0$ & $1$ & $1$ & $0$ & $\boldsymbol{0}$ & $\mu_6$\\
          $M_7$ & $x + \overline{y} + \overline{z} + \overline{w}$ & $0$ & $1$ & $1$ & $1$ & $\boldsymbol{1}$ & $\mu_7$\\
          $M_8$ & $\overline{x} + y + z + w$ & $1$ & $0$ & $0$ & $0$ & $\boldsymbol{0}$ & $\mu_8$\\
          $M_9$ & $\overline{x} + y + z + \overline{w}$ & $1$ & $0$ & $0$ & $1$ & $\boldsymbol{1}$ & $\mu_9$\\
          $M_{10}$ & $\overline{x} + y + \overline{z} + w$ & $1$ & $0$ & $1$ & $0$ & $\boldsymbol{1}$ & $\mu_{10}$\\
          $M_{11}$ & $\overline{x} + y + \overline{z} + \overline{w}$ & $1$ & $0$ & $1$ & $1$ & $\boldsymbol{0}$ & $\mu_{11}$\\
          $M_{12}$ & $\overline{x} + \overline{y} + z + w$ & $1$ & $1$ & $0$ & $0$ & $\boldsymbol{0}$ & $\mu_{12}$\\
          $M_{13}$ & $\overline{x} + \overline{y} + z + \overline{w}$ & $1$ & $1$ & $0$ & $1$ & $\boldsymbol{0}$ & $\mu_{13}$\\
          $M_{14}$ & $\overline{x} + \overline{y} + \overline{z} + w$ & $1$ & $1$ & $1$ & $0$ & $\boldsymbol{0}$ & $\mu_{14}$\\
          $M_{15}$ & $\overline{x} + \overline{y} + \overline{z} + \overline{w}$ & $1$ & $1$ & $1$ & $1$ & $\boldsymbol{1}$ & $\mu_{15}$\\
        \end{tabular}
    }
  \end{tabularx}
  \caption{Funzione Booleana a $4$ variabili associata alla stringa $1010010101100001_2$}
  \label{tab:funzione_booleana_associata_stringa}
\end{table}

\subsection{Codifica dei termini minimi (\emph{minterm})}
\label{sec:codifica_minterm}
Se nella tavola di verità della funzione $f$ considerata si pone in evidenza la codifica dei termini minimi si ottiene:

\begin{table}[H]
  \centering
  \noindent
  \begin{tabular}{c|c||cccc||c|c}
    $ $ & $ $ & $x$ & $y$ & $z$ & $w$ & $\boldsymbol{f}$\\
    \hline
    $m_0$ & $\overline{x}\overline{y}\overline{z}\overline{w}$ & $0$ & $0$ & $0$ & $0$ & $\boldsymbol{1}$ & $\mu_0$\\
    $m_1$ & $\overline{x}\overline{y}\overline{z}w$ & $0$ & $0$ & $0$ & $1$ & $\boldsymbol{0}$ & $\mu_1$\\
    $m_2$ & $\overline{x}\overline{y}z\overline{w}$ & $0$ & $0$ & $1$ & $0$ & $\boldsymbol{1}$ & $\mu_2$\\
    $m_3$ & $\overline{x}\overline{y}zw$ & $0$ & $0$ & $1$ & $1$ & $\boldsymbol{0}$ & $\mu_3$\\
    $m_4$ & $\overline{x}y\overline{z}\overline{w}$ & $0$ & $1$ & $0$ & $0$ & $\boldsymbol{0}$ & $\mu_4$\\
    $m_5$ & $\overline{x}y\overline{z}w$ & $0$ & $1$ & $0$ & $1$ & $\boldsymbol{1}$ & $\mu_5$\\
    $m_6$ & $\overline{x}yz\overline{w}$ & $0$ & $1$ & $1$ & $0$ & $\boldsymbol{0}$ & $\mu_6$\\
    $m_7$ & $\overline{x}yzw$ & $0$ & $1$ & $1$ & $1$ & $\boldsymbol{1}$ & $\mu_7$\\
    $m_8$ & $x\overline{y}\overline{z}\overline{w}$ & $1$ & $0$ & $0$ & $0$ & $\boldsymbol{0}$ & $\mu_8$\\
    $m_9$ & $x\overline{y}\overline{z}w$ & $1$ & $0$ & $0$ & $1$ & $\boldsymbol{1}$ & $\mu_9$\\
    $m_{10}$ & $x\overline{y}z\overline{w}$ & $1$ & $0$ & $1$ & $0$ & $\boldsymbol{1}$ & $\mu_{10}$\\
    $m_{11}$ & $x\overline{y}zw$ & $1$ & $0$ & $1$ & $1$ & $\boldsymbol{0}$ & $\mu_{11}$\\
    $m_{12}$ & $xy\overline{z}\overline{w}$ & $1$ & $1$ & $0$ & $0$ & $\boldsymbol{0}$ & $\mu_{12}$\\
    $m_{13}$ & $xy\overline{z}w$ & $1$ & $1$ & $0$ & $1$ & $\boldsymbol{0}$ & $\mu_{13}$\\
    $m_{14}$ & $xyz\overline{w}$ & $1$ & $1$ & $1$ & $0$ & $\boldsymbol{0}$ & $\mu_{14}$\\
    $m_{15}$ & $xyzw$ & $1$ & $1$ & $1$ & $1$ & $\boldsymbol{1}$ & $\mu_{15}$\\
  \end{tabular}
  \caption{Codifica dei termini minimi}
  \label{tab:codifica_termini_minimi}
\end{table}

\noindent
Pertanto, se si codificano le quaterne d’ingresso associate a ciascun termine minimo con il corrispondente intero rappresentato in notazione posizionale in base $2$, è possibile indicare i termini minimi che compongono la sommatoria di prodotti usando gli interi compresi tra $0$ e $2^{4} - 1$, come illustrato di seguito:
\[f(x, y, z, w) = \underset{i = 0}{\overset{2^4 - 1}{\sum}} \mu_i \cdot m_i = \underset{i: \mu_i = 1}{\sum} m_i\]
dove \(\mu_i\) è il valore assunto dalla funzione in corrispondenza del termine minimo \(m_i\) e \(0 \leq i \leq 2^n -1\).\\
Nel caso analizzato, si ha \(m_0 = \overline{x}\overline{y}\overline{z}\overline{w}, m_2 = \overline{x}\overline{y}z\overline{w}, m_5 = \overline{x}y\overline{z}w, m_7 = \overline{x}yzw, m_9 = x\overline{y}\overline{z}w, m_{10} = x\overline{y}z\overline{w}, m_{15} = xyzw, \mu_0 = \mu_2 = \mu_5 = \mu_7 = \mu_9 = \mu_{10} = \mu_{15} = 1, \mu_1 = \mu_3 = \mu_4 = \mu_6 = \mu_8 = \mu_{11} = \mu_{12} = \mu_{13} = \mu_{14} = 0\).\\
Per cui si perviene al risultato seguente
\[f(x, y, z, w) = \sum_{i \in \{0, 2, 5, 7, 9, 10, 15\}} m_i = m_0 + m_2 + m_5 + m_7 + m_9 + m_{10} + m_{15}\]
Quindi l'espressione dei \emph{minterm} è:
\[f(x,y,z,w) = \overline{x}\overline{y}\overline{z}\overline{w} + \overline{x}\overline{y}z\overline{w} + \overline{x}y\overline{z}w + \overline{x}yzw + x\overline{y}\overline{z}w + x\overline{y}z\overline{w} + xyzw\]
poiché \(0, 2, 5, 7, 9, 10\) e \(15\) sono le codifiche in base \(2\) di \(0000\), \(0010\), \(0101\), \(0111\), \(1001\), \(1010\) e \(1111\).

\vspace{1em}
\subsection{Codifica dei termini massimi (\emph{maxterm})}
\label{sec:codifica_maxterm}
Analogamente, procedendo per dualità, se nella tavola di verità della funzione $f$ considerata si pone in evidenza la codifica dei termini massimi si ottiene:

\begin{table}[H]
  \centering
  \noindent
  \begin{tabular}{c|c||cccc||c|c}
    $ $ & $ $ & $x$ & $y$ & $z$ & $w$ & $\boldsymbol{f}$\\
    \hline
    $M_0$ & $x + y + z + w$ & $0$ & $0$ & $0$ & $0$ & $\boldsymbol{1}$ & $\mu_0$\\
    $M_1$ & $x + y + z + \overline{w}$ & $0$ & $0$ & $0$ & $1$ & $\boldsymbol{0}$ & $\mu_1$\\
    $M_2$ & $x + y + \overline{z} + w$ & $0$ & $0$ & $1$ & $0$ & $\boldsymbol{1}$ & $\mu_2$\\
    $M_3$ & $x + y + \overline{z} + \overline{w}$ & $0$ & $0$ & $1$ & $1$ & $\boldsymbol{0}$ & $\mu_3$\\
    $M_4$ & $x + \overline{y} + z + w$ & $0$ & $1$ & $0$ & $0$ & $\boldsymbol{0}$ & $\mu_4$\\
    $M_5$ & $x + \overline{y} + z + \overline{w}$ & $0$ & $1$ & $0$ & $1$ & $\boldsymbol{1}$ & $\mu_5$\\
    $M_6$ & $x + \overline{y} + \overline{z} + w$ & $0$ & $1$ & $1$ & $0$ & $\boldsymbol{0}$ & $\mu_6$\\
    $M_7$ & $x + \overline{y} + \overline{z} + \overline{w}$ & $0$ & $1$ & $1$ & $1$ & $\boldsymbol{1}$ & $\mu_7$\\
    $M_8$ & $\overline{x} + y + z + w$ & $1$ & $0$ & $0$ & $0$ & $\boldsymbol{0}$ & $\mu_8$\\
    $M_9$ & $\overline{x} + y + z + \overline{w}$ & $1$ & $0$ & $0$ & $1$ & $\boldsymbol{1}$ & $\mu_9$\\
    $M_{10}$ & $\overline{x} + y + \overline{z} + w$ & $1$ & $0$ & $1$ & $0$ & $\boldsymbol{1}$ & $\mu_{10}$\\
    $M_{11}$ & $\overline{x} + y + \overline{z} + \overline{w}$ & $1$ & $0$ & $1$ & $1$ & $\boldsymbol{0}$ & $\mu_{11}$\\
    $M_{12}$ & $\overline{x} + \overline{y} + z + w$ & $1$ & $1$ & $0$ & $0$ & $\boldsymbol{0}$ & $\mu_{12}$\\
    $M_{13}$ & $\overline{x} + \overline{y} + z + \overline{w}$ & $1$ & $1$ & $0$ & $1$ & $\boldsymbol{0}$ & $\mu_{13}$\\
    $M_{14}$ & $\overline{x} + \overline{y} + \overline{z} + w$ & $1$ & $1$ & $1$ & $0$ & $\boldsymbol{0}$ & $\mu_{14}$\\
    $M_{15}$ & $\overline{x} + \overline{y} + \overline{z} + \overline{w}$ & $1$ & $1$ & $1$ & $1$ & $\boldsymbol{1}$ & $\mu_{15}$\\
  \end{tabular}
  \caption{Codifica dei termini massimi}
  \label{tab:codifica_termini_massimi}
\end{table}

\noindent
Analogamente a quanto già esposto, se ora si codificano le quaterne d’ingresso associate a ciascun termine massimo con il corrispondente intero rappresentato in notazione posizionale in base $2$, è possibile indicare i termini massimi che compongono il prodotto di somme usando gli interi compresi tra $0$ e $2^{4} - 1$, come illustrato di seguito:
\[f(x, y, z, w) = \underset{i = 0}{\overset{2^4 - 1}{\prod}} \mu_i \cdot M_i = \underset{i: \mu_i = 1}{\prod} M_i\]
dove \(\mu_i\) è il valore assunto dalla funzione in corrispondenza del termine massimo \(M_i\) e \(0 \leq i \leq 2^n -1\).\\
Nel caso analizzato, si ha \(M_1 = x + y + z + \overline{w}, M_3 = x + y + \overline{z} + \overline{w}, M_4 = x + \overline{y} + z + w, M_6 = x + \overline{y} + \overline{z} + w, M_8 = \overline{x} + y + z + w, M_{11} = \overline{x} + y + \overline{z} + \overline{w}, M_{12} = \overline{x} + \overline{y} + z + w, M_{13} \overline{x} + \overline{y} + z + \overline{w}, M_{14} = \overline{x} + \overline{y} + \overline{z} + w, \mu_0 = \mu_2 = \mu_5 = \mu_7 = \mu_9 = \mu_{10} = \mu_{15} = 1, \mu_1 = \mu_3 = \mu_4 = \mu_6 = \mu_8 = \mu_{11} = \mu_{12} = \mu_{13} = \mu_{14} = 0\).\\
Per cui si perviene al risultato seguente
\[f(x, y, z, w) = \prod_{i \in \{1, 3, 4, 6, 8, 11, 12, 13, 14\}} M_i = M_1 \cdot M_3 \cdot M_4 \cdot M_8 \cdot M_{11} \cdot M_{12} \cdot M_{13} \cdot M_{14}\]
Quindi l'espressione dei \emph{maxterm} è:
\begin{align*}
  f(x,y,z,w) = \left(x + y + z + \overline{w}\right) \cdot \left(x + y + \overline{z} + \overline{w}\right) \cdot \left(x + \overline{y} + z + w\right) \cdot \left(x + \overline{y} + \overline{z} + w\right) \cdot \left(\overline{x} + y + z + w\right)\\
  \cdot \left(\overline{x} + y + \overline{z} + \overline{w}\right) \cdot \left(\overline{x} + \overline{y} + z + w\right) \cdot \left(\overline{x} + \overline{y} + z + \overline{w}\right) \cdot \left(\overline{x} + \overline{y} + \overline{z} + w\right)
\end{align*}
poiché \(1, 3, 4, 6, 8, 11, 12, 13\) e \(14\) sono le codifiche in base \(2\) di \(0001\), \(0011\), \(0100\), \(0110\), \(1000\), \(1011\), \(1100\), \(1101\) e \(1110\).

\newpage
\section{Semplificazione dell'espressione Booleana}
Di seguito si espongono i $3$ diversi procedimenti di semplificazione dell'espressione Booleana precedentemente ottenuta, ricondotta alla forma minima tramite l'\emph{applicazione delle relazioni fondamentali dell'Algebra Booleana} (assiomi e teoremi), tramite le \emph{mappe di Karnaugh} e attraverso il \emph{metodo tabellare di Quine - Mc Cluskey}.

\vspace{1em}
\subsection{Semplificazione per via algebrica}
\label{sec:semplificazione_algebrica}
Si procede, ora, alla semplificazione delle espressioni ottenute per via diretta, facendo uso degli assiomi A1-A7 e dei teoremi T1-T10 dell’Algebra Booleana.

\vspace{1em}
\subsubsection{Semplificazione dei \emph{minterm}}
\begin{eqnarray*}
  f(x,y,z,w) &=& \overline{x}\overline{y}\overline{z}\overline{w} + \overline{x}\overline{y}z\overline{w} + \overline{x}y\overline{z}w + \overline{x}yzw + x\overline{y}\overline{z}w + x\overline{y}z\overline{w} + xyzw\\
  &\overset{\left(A4 \text{ e } A5\right)}{=}& \left(\overline{x}\overline{y}\overline{z}\overline{w} + \overline{x}\overline{y}z\overline{w}\right) + \left(\overline{x}y\overline{z}w + \overline{x}yzw\right) + x\overline{y}\overline{z}w + x\overline{y}z\overline{w} + xyzw\\
  &\overset{\left(T9\right)}{=}& \overline{x}\overline{y}\overline{w} + \overline{x}yw + x\overline{y}\overline{z}w + x\overline{y}z\overline{w} + xyzw\\
  &\overset{\left(A4\right)}{=}& \overline{x}\overline{y}\overline{w} + x\overline{y}z\overline{w} + \overline{x}yw + xyzw + x\overline{y}\overline{z}w\\
  &\overset{\left(A4 \text{ e } A6\right)}{=}& \overline{y}\overline{w} \cdot \left(\overline{x} + xz\right) + yw \cdot \left(\overline{x} + xz\right) + x\overline{y}\overline{z}w\\
  &\overset{\left(T5\right)}{=}& \overline{y}\overline{w} \cdot \left(\overline{x} + z\right) + yw \cdot \left(\overline{x} + z\right) + x\overline{y}\overline{z}w\\
  &\overset{\left(A4 \text{ e } A6\right)}{=}& \overline{x}\overline{y}\overline{w} + \overline{y}z\overline{w} + \overline{x}yw + yzw + x\overline{y}\overline{z}w\\
\end{eqnarray*}

\vspace{1em}
\subsubsection{Semplificazione dei \emph{maxterm}}
\begin{eqnarray*}
  f(x,y,z,w) \hspace{-2.5em} &=& \left(x + y + z + \overline{w}\right) \cdot \left(x + y + \overline{z} + \overline{w}\right) \cdot \left(x + \overline{y} + z + w\right) \cdot \left(x + \overline{y} + \overline{z} + w\right) \cdot \left(\overline{x} + y + z + w\right)\\
  & & \cdot \left(\overline{x} + y + \overline{z} + \overline{w}\right) \cdot \left(\overline{x} + \overline{y} + z + w\right) \cdot \left(\overline{x} + \overline{y} + z + \overline{w}\right) \cdot \left(\overline{x} + \overline{y} + \overline{z} + w\right)\\
  &\overset{\left(A4 \text{ e } A5\right)}{=}& \left[\left(x + y + z + \overline{w}\right) \cdot \left(x + y + \overline{z} + \overline{w}\right) \right] \cdot \left[\left(x + \overline{y} + z + w\right) \cdot \left(x + \overline{y} + \overline{z} + w\right) \right] \cdot \left[\left(\overline{x} + y + z + w\right)\right.\\
  & & \cdot \left.\left(\overline{x} + \overline{y} + z + w\right) \right] \cdot \left(\overline{x} + y + \overline{z} + \overline{w}\right) \cdot \left(\overline{x} + \overline{y} + z + \overline{w}\right) \cdot \left(\overline{x} + \overline{y} + \overline{z} + w\right)\\
  &\overset{\left(T9\right)}{=}& \left(x + y + \overline{w}\right) \cdot \left(x + \overline{y} + w\right) \cdot \left(\overline{x} + z + w\right) \cdot \left(\overline{x} + y + \overline{z} + \overline{w}\right) \cdot \left(\overline{x} + \overline{y} + z + \overline{w}\right) \cdot \left(\overline{x} + \overline{y} + \overline{z} + w\right)\\
  &\overset{\left(A4\right)}{=}& \left(x + \overline{y} + w\right) \cdot \left(\overline{x} + \overline{y} + \overline{z} + w\right) \cdot \left(\overline{x} + z + w\right) \cdot \left(\overline{x} + \overline{y} + z + \overline{w}\right) \cdot \left(x + y + \overline{w}\right) \cdot \left(\overline{x} + y + \overline{z} + \overline{w}\right)\\
  &\overset{\left(A4 \text{ e } A6\right)}{=}& \left[\overline{y} + w + x \cdot \left(\overline{x} + \overline{z}\right) \right] \cdot \left[\overline{x} + z + w \cdot \left( \overline{y} + \overline{w}\right) \right] \cdot \left[y + \overline{w} + x \cdot \left(\overline{x} + \overline{z} \right)\right]\\
  &\overset{\left(A4 \text{ e } T5\right)}{=}& \left(\overline{y} + w + x\overline{z}\right) \cdot \left(\overline{x} + z + \overline{y}w \right) \cdot \left(y + \overline{w} + x\overline{z}\right)\\
  &\overset{\left(A4 \text{ e } T3\right)}{=}& \left(\overline{\overline{\overline{y} + w}} + x\overline{z}\right) \cdot \left(\overline{y}w + \overline{\overline{\overline{x} + z}} \right) \cdot \left(\overline{\overline{y + \overline{w}}} + x\overline{z}\right)\\
  &\overset{\left(A5 \text{ e } T7\right)}{=}& \left[\left(\overline{y\overline{w}} + x\overline{z}\right) \cdot \left(\overline{y}w + \overline{x\overline{z}} \right) \right] \cdot \left(\overline{\overline{y}w} + x\overline{z}\right)\\
  &\overset{\left(A4 \text{ e } T10\right)}{=}& \left(x\overline{y}\overline{z}w + \overline{x\overline{z}} \cdot \overline{y\overline{w}}\right) \cdot \left(\overline{\overline{y}w} + x\overline{z}\right)\\
  &\overset{\left(A6 \text{ e } A7 \text{ e } T1\right)}{=}& x\overline{y}\overline{z}w + \overline{x\overline{z}} \cdot \overline{y\overline{w}} \cdot \overline{\overline{y}w}\\
  &\overset{\left(A5 \text{ e } T7\right)}{=}& x\overline{y}\overline{z}w + \left(\overline{x} + z\right) \cdot \left[\left(\overline{y} + w\right) \cdot \left(y + \overline{w}\right)\right]\\
  &\overset{\left(A4 \text{ e } T10\right)}{=}& x\overline{y}\overline{z}w + \left(\overline{x} + z\right) \cdot \left(yw + \overline{y}\overline{w}\right)\\
  &\overset{\left(A4 \text{ e } A6\right)}{=}& \overline{x}\overline{y}\overline{w} + \overline{y}z\overline{w} + \overline{x}yw + yzw + x\overline{y}\overline{z}w\\
\end{eqnarray*}

\vspace{1em}
\subsubsection{Equivalenza delle forme canoniche}
Naturalmente è noto che la \emph{somma di prodotti} e il \emph{prodotto di somme} ottenuti rappresentano la medesima funzione; inoltre, tramite una serie di procedimenti algebrici precedentemente esposti, è stato dimostrato che le due forme canoniche si equivalgnono non solo dal punto di vista logico, ma anche dal punto di vista formale: infatti, le due espressioni semplificate, ottenute a partire da \emph{minterm} e \emph{maxterm} conducono alla stessa formula minima:
\[f(x,y,z,w) = \overline{x}\overline{y}\overline{w} + \overline{y}z\overline{w} + \overline{x}yw + yzw + x\overline{y}\overline{z}w\]

\vspace{1em}
\subsubsection{Mappa di Karnaugh}
\label{sec:semplificazione_mappa_karnaugh}
A partire dai \emph{minterm}, si espone di seguito la semplificazione della funzione considerata tramite la mappa di Karnaugh:

% Contenitore per immagini
\begin{figure}[H]
    \centering
    \begin{Karnaugh}
        \contingut{1,0,0,0,0,1,1,0,1,0,1,0,0,1,0,1}
        \implicantdaltbaix{0}{8}{violet}
        \implicantcostats{8}{10}{orange}
        \implicant{5}{13}{blue}
        \implicant{13}{15}{yellow}
        \implicant{6}{6}{green}
    \end{Karnaugh}
    \vspace{-1em}
    \caption{Mappa di Karnaugh per la funzione}
    \label{fig:mappa_karnaugh_funzione}
\end{figure}

\noindent
Ricordando che, all'interno della mappa di Karnaugh, due caselle adiacenti differiscono per il valore di una sola variabile, è facile capire come il prodotto delle variabili comuni implica entrambi i termini minimi associati alle caselle considerate. Da ciò segue il risultato:
\[f(x,y,z,w) = \overline{x}\overline{y}\overline{w} + \overline{y}z\overline{w} + \overline{x}yw + yzw + x\overline{y}\overline{z}w\]
Ancora una volta, la funzione ottenuta dalla semplificazione tramite mappa di Karnaugh corrisponde alla funzione ottenuta nei passi precedenti.

\newpage
\subsubsection{Metodo tabellare di Quine-Mc Cluskey}
\label{sec:semplificazione_tabella_quine-mc_cluskey}
A partire dai \emph{minterm}, si espone di seguito la semplificazione della funzione considerata tramite il metodo tabellare di Quine-Mc Cluskey:

\begin{table}[H]
    \centering
    \setlength{\tabcolsep}{2pt}
    \begin{tabularx}{\textwidth}{ccc}
    {
        \setlength{\tabcolsep}{2pt}
        \begin{tabular}{c|c|cccc|cl}
          $ $ & $ $ & $x$ & $y$ & $z$ & $w$ & $f$\\
          \hline
          $0$ & $\overline{x}\overline{y}\overline{z}\overline{w}$ & $0$ & $0$ & $0$ & $0$ & $1$ & $\textcolor{red}{\ast}$\\
          $1$ & $\overline{x}\overline{y}\overline{z}w$ & $0$ & $0$ & $0$ & $1$ & $0$ & $$\\
          $2$ & $\overline{x}\overline{y}z\overline{w}$ & $0$ & $0$ & $1$ & $0$ & $1$ & $\textcolor{red}{\ast}$\\
          $3$ & $\overline{x}\overline{y}zw$ & $0$ & $0$ & $1$ & $1$ & $0$ & $$\\
          $4$ & $\overline{x}y\overline{z}\overline{w}$ & $0$ & $1$ & $0$ & $0$ & $0$ & $$\\
          $5$ & $\overline{x}y\overline{z}w$ & $0$ & $1$ & $0$ & $1$ & $1$ & $\textcolor{blue}{\ast}$\\
          $6$ & $\overline{x}yz\overline{w}$ & $0$ & $1$ & $1$ & $0$ & $0$ & $$\\
          $7$ & $\overline{x}yzw$ & $0$ & $1$ & $1$ & $1$ & $1$ & $\textcolor{blue}{\ast}$\\
          $8$ & $x\overline{y}\overline{z}\overline{w}$ & $1$ & $0$ & $0$ & $0$ & $0$ & $ $\\
          $9$ & $x\overline{y}\overline{z}w$ & $1$ & $0$ & $0$ & $1$ & $1$ & $\textcolor{magenta}{\ast}$\\
          $10$ & $x\overline{y}z\overline{w}$ & $1$ & $0$ & $1$ & $0$ & $1$ & $\textcolor{orange}{\ast}$\\
          $11$ & $x\overline{y}zw$ & $1$ & $0$ & $1$ & $1$ & $0$ & $ $\\
          $12$ & $xy\overline{z}\overline{w}$ & $1$ & $1$ & $0$ & $0$ & $0$ & $ $\\
          $13$ & $xy\overline{z}w$ & $1$ & $1$ & $0$ & $1$ & $0$ & $ $\\
          $14$ & $xyz\overline{w}$ & $1$ & $1$ & $1$ & $0$ & $0$ & $ $\\
          $15$ & $xyzw$ & $1$ & $1$ & $1$ & $1$ & $1$ & $\textcolor{violet}{\ast}$\\
        \end{tabular}
    }
    &
    {
        \setlength{\tabcolsep}{2pt}
        \begin{tabular}{c|c|c|cccc|cl}
          \textbf{Livello} & $ $ & $ $ & $x$ & $y$ & $z$ & $w$ & $f$\\
          \hline
          $0$ & $0$ & $\overline{x}\overline{y}\overline{z}\overline{w}$ & $0$ & $0$ & $0$ & $0$ & $1$ & $\textcolor{red}{\ast}$\\
          \hline
          \multirow{4}{*}{$1$} & $1$ & $\overline{x}\overline{y}\overline{z}w$ & $0$ & $0$ & $0$ & $1$ & $0$ & $$\\
          & $2$ & $\overline{x}\overline{y}z\overline{w}$ & $0$ & $0$ & $1$ & $0$ & $1$ & $\textcolor{red}{\ast}$\\
          & $4$ & $\overline{x}y\overline{z}\overline{w}$ & $0$ & $1$ & $0$ & $0$ & $0$ & $$\\
          & $8$ & $x\overline{y}\overline{z}\overline{w}$ & $1$ & $0$ & $0$ & $0$ & $0$ & $ $\\
          \hline
          \multirow{6}{*}{$2$} & $3$ & $\overline{x}\overline{y}zw$ & $0$ & $0$ & $1$ & $1$ & $0$ & $$\\
          & $5$ & $\overline{x}y\overline{z}w$ & $0$ & $1$ & $0$ & $1$ & $1$ & $\textcolor{blue}{\ast}$\\
          & $6$ & $\overline{x}yz\overline{w}$ & $0$ & $1$ & $1$ & $0$ & $0$ & $$\\
          & $9$ & $x\overline{y}\overline{z}w$ & $1$ & $0$ & $0$ & $1$ & $1$ & $\textcolor{magenta}{\ast}$\\
          & $10$ & $x\overline{y}z\overline{w}$ & $1$ & $0$ & $1$ & $0$ & $1$ & $\textcolor{orange}{\ast}$\\
          & $12$ & $xy\overline{z}\overline{w}$ & $1$ & $1$ & $0$ & $0$ & $0$ & $ $\\
          \hline
          \multirow{4}{*}{$3$} & $7$ & $\overline{x}yzw$ & $0$ & $1$ & $1$ & $1$ & $1$ & $\textcolor{blue}{\ast}$\\
          & $11$ & $x\overline{y}zw$ & $1$ & $0$ & $1$ & $1$ & $0$ & $ $\\
          & $13$ & $xy\overline{z}w$ & $1$ & $1$ & $0$ & $1$ & $0$ & $ $\\
          & $14$ & $xyz\overline{w}$ & $1$ & $1$ & $1$ & $0$ & $0$ & $ $\\
          \hline
          $4$ & $15$ & $xyzw$ & $1$ & $1$ & $1$ & $1$ & $1$ & $\textcolor{violet}{\ast}$\\
        \end{tabular}
    }
    &
    {
        \setlength{\tabcolsep}{2pt}
        \begin{tabular}{c|c|c|cccc|cl}
          \textbf{Livello} & $ $ & $ $ & $x$ & $y$ & $z$ & $w$ & $f$\\
          \hline
          $0$ & $0$ & $\overline{x}\overline{y}\overline{z}\overline{w}$ & $0$ & $0$ & $0$ & $0$ & $1$ & $\textcolor{red}{\ast}$\\
          \hline
          $1$ & $2$ & $\overline{x}\overline{y}z\overline{w}$ & $0$ & $0$ & $1$ & $0$ & $1$ & $\textcolor{red}{\ast}$\\
          \hline
          \multirow{3}{*}{$2$} & $5$ & $\overline{x}y\overline{z}w$ & $0$ & $1$ & $0$ & $1$ & $1$ & $\textcolor{blue}{\ast}$\\
          & $9$ & $x\overline{y}\overline{z}w$ & $1$ & $0$ & $0$ & $1$ & $1$ & $\textcolor{magenta}{\ast}$\\
          & $10$ & $x\overline{y}z\overline{w}$ & $1$ & $0$ & $1$ & $0$ & $1$ & $\textcolor{orange}{\ast}$\\
          \hline
          $3$ & $7$ & $\overline{x}yzw$ & $0$ & $1$ & $1$ & $1$ & $1$ & $\textcolor{blue}{\ast}$\\
          \hline
          $4$ & $15$ & $xyzw$ & $1$ & $1$ & $1$ & $1$ & $1$ & $\textcolor{violet}{\ast}$\\
        \end{tabular}
    }
    \end{tabularx}
    \caption{Trasformazione della tavola di verità della funzione logica \(f\) per ottenere la tabella di Quine-Mc Cuskey}
    \label{tab:tavola_verita_Quine_Mc_Cuskey}
\end{table}

\noindent
La tabella di Quine-Mc Cuskey così ottenuta può essere rappresentata, in maniera più, schematica, nella tabella seguente

\begin{table}[H]
    \centering
    \setlength{\tabcolsep}{3.5pt}
    \begin{tabular}{c|c|c|c}
      $ $ & \text{Livello} & \text{Numero} & \text{Termine minimo}\\
      \hline
      $\sqrt{}$ & $0$ & $0$ & $0000$\\
      \hline
      $\sqrt{}$ & $1$ & $2$ & $0010$\\
      \hline
      $\sqrt{}$ & $2$ & $5$ & $0101$\\
      $\sqrt{}$ & $2$ & $9$ & $1001$\\
      $\sqrt{}$ & $2$ & $10$ & $1010$\\
      \hline
      $\sqrt{}$ & $3$ & $7$ & $0111$\\
      \hline
      $\sqrt{}$ & $4$ & $15$ & $1111$\\
    \end{tabular}
    \caption{Divisione dei termini minimi in livelli e riordinamento degli stessi}
    \label{tab:divisione_minterm_livello}
\end{table}

\noindent
Si può procedere, ora, alle semplificazioni opportune, partendo dal primo termine minimo del primo livello disponibile e confrontando tutti i termini minimi del livello $k$ con tutti quelli del livello $k + 1$, semplificando tra loro i termini che differiscono per un solo bit.\\
Tale procedimento genera la Tabella \ref{tab:tabella_seplificazione}, in cui le semplificazioni avvenute si indicano con una lineetta, mentre gli implicanti vengono contraddistinti con i numeri dei termini minimi che li hanno generati:

\begin{table}[H]
    \centering
    \setlength{\tabcolsep}{3.5pt}
    \begin{tabular}{c|c|cccc|c}
      $\sqrt{}$ & $0,2$ & $0$ & $0$ & $-$ & $0$ & \text{A}\\
      \hline
      $\sqrt{}$ & $2,10$ & $-$ & $0$ & $1$ & $0$ & \text{B}\\
      \hline
      $\sqrt{}$ & $5,7$ & $0$ & $1$ & $-$ & $1$ & \text{C}\\
      \hline
      $\sqrt{}$ & $7,15$ & $-$ & $1$ & $1$ & $1$ & \text{D}\\
    \end{tabular}
    \caption{Tabella di semplificazione}
    \label{tab:tabella_seplificazione}
\end{table}

\noindent
Da tale semplificazione è evidente che il termine minimo $1001$ non ha portato ad alcuna semplificazione, pertanto si contrassegna anch'esso con la lettera E, proseguendo la progressione alfabetica adottata in fase di semplificazione.\\
Si procede, infine, alla costruzione del cosiddetto \emph{reticolo del metodo}, avente i termini minimi sulle colonne e gli implicanti primi sulle righe. Su ogni riga, cioè in corrispondenza di ciascun implicante, si contrassegnano opportunamente i termini minimi implicati. Per la funzione considerata si ottiene

% Contenitore per immagini
\begin{figure}[H]
    \centering
    \begin{tikzpicture}
    \tikzset{dot/.style={fill=black,circle}}

    \foreach \l [count=\y] in {E,...,A} {
      \draw (0,\y) -- (8,\y);
      \node at (-0.5,\y){\l};
    }

    \foreach \x [count=\i] in {0,2,5,7,9,10,15} {
        \draw (\i,0) -- (\i,6);
        \node at (\i,-0.5){\x};
    }

    % A
    \draw node[fill,circle,inner sep=0pt,minimum size=5pt] at (1,5){};
    \draw node[fill,circle,inner sep=0pt,minimum size=5pt] at (2,5){};

    % B
    \draw node[fill,circle,inner sep=0pt,minimum size=5pt] at (2,4){};
    \draw node[fill,circle,inner sep=0pt,minimum size=5pt] at (6,4){};

    % C
    \draw node[fill,circle,inner sep=0pt,minimum size=5pt] at (3,3){};
    \draw node[fill,circle,inner sep=0pt,minimum size=5pt] at (4,3){};

    % D
    \draw node[fill,circle,inner sep=0pt,minimum size=5pt] at (4,2){};
    \draw node[fill,circle,inner sep=0pt,minimum size=5pt] at (7,2){};

    % E
    \draw node[fill,circle,inner sep=0pt,minimum size=5pt] at (5,1){};
    \end{tikzpicture}
    \caption{Reticolo di semplificazione}
    \label{fig:reticolo_semplificazione}
\end{figure}

\noindent
Dall’esame del reticolo in Figura \ref{fig:reticolo_semplificazione} si individuano facilmente i termini minimi che sono implicati da un unico implicante primo. Nel caso della funzione considerata, è facile osservare come gli implicanti primi A, B, C, D ed E sono tutti e cinque essenziali; l’espressione semplificata è allora
\[f = A + B + C + D + E = \overline{x}\overline{y}\overline{w} + \overline{y}z\overline{w} + \overline{x}yw + yzw + x\overline{y}\overline{z}w\]
Ancora una volta, la funzione ottenuta dalla semplificazione tramite il metodo tabellare
di Quine-Mc Cluskey corrisponde alla funzione ottenuta nei passi precedenti.

\newpage
\section{Interpretazione circuitale}
Di seguito si espone lo schema logico, basato su porte AND, OR e NOT, delle seguenti funzioni:
\begin{itemize}
  \item della funzione ottenuta dai \emph{minterm}, nella sezione §\ref{sec:codifica_minterm};
  \item della funzione ottenuta dai \emph{maxterm}, nella sezione §\ref{sec:codifica_maxterm};
  \item della funzione semplificata, ottenuta nelle sezioni §\ref{sec:semplificazione_algebrica}, §\ref{sec:semplificazione_mappa_karnaugh} e §\ref{sec:semplificazione_tabella_quine-mc_cluskey}.
\end{itemize}

\vspace{1em}
\subsection{Funzione ottenuta dai \emph{minterm}}

\begin{figure}[H]
    \centering
    \scalebox{1.25}{
        \centering
        \begin{tikzpicture}
          % Variabili
          \coordinate (x) at (0,0);
          \coordinate (y) at (1.5,0);
          \coordinate (z) at (3,0);
          \coordinate (w) at (4.5,0);

          % Porte logiche
          \node [not gate US, draw, rotate=270] (notx) at ([xshift=6mm,yshift=-8mm]x){};
          \node [not gate US, draw, rotate=270] (noty) at ([xshift=6mm,yshift=-8mm]y){};
          \node [not gate US, draw, rotate=270] (notz) at ([xshift=6mm,yshift=-8mm]z){};
          \node [not gate US, draw, rotate=270] (notw) at ([xshift=6mm,yshift=-8mm]w){};

          \node [and gate US, draw, logic gate inputs=nnnn] (and1) at (6.5,-2.5){$\overline{x}\overline{y}\overline{z}\overline{w}$};
          \node [and gate US, draw, below=of and1, yshift=3.5mm, logic gate inputs=nnnn] (and2){$\overline{x}\overline{y}z\overline{w}$};
          \node [and gate US, draw, below=of and2, yshift=3.5mm, logic gate inputs=nnnn] (and3){$\overline{x}y\overline{z}w$};
          \node [and gate US, draw, below=of and3, yshift=3.5mm, logic gate inputs=nnnn] (and4){$\overline{x}yzw$};
          \node [and gate US, draw, below=of and4, yshift=3.5mm, logic gate inputs=nnnn] (and5){$x\overline{y}\overline{z}w$};
          \node [and gate US, draw, below=of and5, yshift=3.5mm, logic gate inputs=nnnn] (and6){$x\overline{y}z\overline{w}$};
          \node [and gate US, draw, below=of and6, yshift=3.5mm, logic gate inputs=nnnn] (and7){$xyzw$};
          \node [or gate US, draw, right=of and4, xshift=5mm, logic gate inputs=nnnnnnn] (or){};

          % INPUT
          \draw (x) ++(0,-14) node[](xend){} -- node[at end, above]{$x$} (x);
          \draw (y) ++(0,-14) -- node[at end, above]{$y$} (y);
          \draw (z) ++(0,-14) -- node[at end, above]{$z$} (z);
          \draw (w) ++(0,-14) -- node[at end, above]{$w$} (w);

          % NOT
          \draw (notx.input) -- ++(0,3mm) -- ([yshift=3mm] notx.input -| x) node [branch]{};
          \draw (notx.output) -- (notx.output |- xend);

          \draw (noty.input) -- ++(0,3mm) -- ([yshift=3mm] noty.input -| y) node [branch]{};
          \draw (noty.output) -- (noty.output |- xend);

          \draw (notz.input) -- ++(0,3mm) -- ([yshift=3mm] notz.input -| z) node [branch]{};
          \draw (notz.output) -- (notz.output |- xend);

          \draw (notw.input) -- ++(0,3mm) -- ([yshift=3mm] notw.input -| w) node [branch]{};
          \draw (notw.output) -- (notw.output |- xend);

          % AND 1
          \draw (and1.input 1) -- (and1.input 1 -| notx.output) node [branch]{};
          \draw (and1.input 2) -- (and1.input 2 -| noty.output) node [branch]{};
          \draw (and1.input 3) -- (and1.input 3 -| notz.output) node [branch]{};
          \draw (and1.input 4) -- (and1.input 4 -| notw.output) node [branch]{};

          % AND 2
          \draw (and2.input 1) -- (and2.input 1 -| notx.output) node [branch]{};
          \draw (and2.input 2) -- (and2.input 2 -| noty.output) node [branch]{};
          \draw (and2.input 3) -- (and2.input 3 -| z) node [branch]{};
          \draw (and2.input 4) -- (and2.input 4 -| notw.output) node [branch]{};

          % AND 3
          \draw (and3.input 1) -- (and3.input 1 -| notx.output) node [branch]{};
          \draw (and3.input 2) -- (and3.input 2 -| y) node [branch]{};
          \draw (and3.input 3) -- (and3.input 3 -| notz.output) node [branch]{};
          \draw (and3.input 4) -- (and3.input 4 -| w) node [branch]{};

          % AND 4
          \draw (and4.input 1) -- (and4.input 1 -| notx.output) node [branch]{};
          \draw (and4.input 2) -- (and4.input 2 -| y) node [branch]{};
          \draw (and4.input 3) -- (and4.input 3 -| z) node [branch]{};
          \draw (and4.input 4) -- (and4.input 4 -| w) node [branch]{};

          % AND 5
          \draw (and5.input 1) -- (and5.input 1 -| x) node [branch]{};
          \draw (and5.input 2) -- (and5.input 2 -| noty.output) node [branch]{};
          \draw (and5.input 3) -- (and5.input 3 -| notz.output) node [branch]{};
          \draw (and5.input 4) -- (and5.input 4 -| w) node [branch]{};

          % AND 6
          \draw (and6.input 1) -- (and6.input 1 -| x) node [branch]{};
          \draw (and6.input 2) -- (and6.input 2 -| noty.output) node [branch]{};
          \draw (and6.input 3) -- (and6.input 3 -| z) node [branch]{};
          \draw (and6.input 4) -- (and6.input 4 -| notw.output) node [branch]{};

          % AND 7
          \draw (and7.input 1) -- (and7.input 1 -| x) node [branch]{};
          \draw (and7.input 2) -- (and7.input 2 -| y) node [branch]{};
          \draw (and7.input 3) -- (and7.input 3 -| z) node [branch]{};
          \draw (and7.input 4) -- (and7.input 4 -| w) node [branch]{};

          % OUTPUT
          \draw (and1.output) -- ([xshift=0.7cm] and1.output) |- (or.input 1);
          \draw (and2.output) -- ([xshift=0.5cm] and2.output) |- (or.input 2);
          \draw (and3.output) -- ([xshift=0.3cm] and3.output) |- (or.input 3);
          \draw (and4.output) -- (and4.output) |- (or.input 4);
          \draw (and5.output) -- ([xshift=0.3cm] and5.output) |- (or.input 5);
          \draw (and6.output) -- ([xshift=0.5cm] and6.output) |- (or.input 6);
          \draw (and7.output) -- ([xshift=0.7cm] and7.output) |- (or.input 7);
          \draw (or.output) -- node [above] {$f$} ([xshift=0.5cm] or.output);
        \end{tikzpicture}
    }
    \caption{Funzione ottenuta dai \emph{minterm}}
    \label{fig:funzione_ottenuta_minterm}
\end{figure}

\vspace{1em}
\subsection{Funzione ottenuta dai \emph{maxterm}}

\begin{figure}[H]
    \centering
    \scalebox{1.25}{
        \centering
        \begin{tikzpicture}
          % Variabili
          \coordinate (x) at (0,0);
          \coordinate (y) at (1.5,0);
          \coordinate (z) at (3,0);
          \coordinate (w) at (4.5,0);

          % Porte logiche
          \node [not gate US, draw, rotate=270] (notx) at ([xshift=6mm,yshift=-8mm]x){};
          \node [not gate US, draw, rotate=270] (noty) at ([xshift=6mm,yshift=-8mm]y){};
          \node [not gate US, draw, rotate=270] (notz) at ([xshift=6mm,yshift=-8mm]z){};
          \node [not gate US, draw, rotate=270] (notw) at ([xshift=6mm,yshift=-8mm]w){};

          \node [or gate US, draw, logic gate inputs=nnnn] (or1) at (6.5,-2.5){};
          \node [or gate US, draw, below=of or1, yshift=1.5mm, logic gate inputs=nnnn] (or2){};
          \node [or gate US, draw, below=of or2, yshift=1.5mm, logic gate inputs=nnnn] (or3){};
          \node [or gate US, draw, below=of or3, yshift=1.5mm, logic gate inputs=nnnn] (or4){};
          \node [or gate US, draw, below=of or4, yshift=1.5mm, logic gate inputs=nnnn] (or5){};
          \node [or gate US, draw, below=of or5, yshift=1.5mm, logic gate inputs=nnnn] (or6){};
          \node [or gate US, draw, below=of or6, yshift=1.5mm, logic gate inputs=nnnn] (or7){};
          \node [or gate US, draw, below=of or7, yshift=1.5mm, logic gate inputs=nnnn] (or8){};
          \node [or gate US, draw, below=of or8, yshift=1.5mm, logic gate inputs=nnnn] (or9){};
          \node [and gate US, draw, right=of or5, xshift=8mm, logic gate inputs=nnnnnnnnn] (and){};

          % INPUT
          \draw (x) ++(0,-17) node[](xend){} -- node[at end, above]{$x$} (x);
          \draw (y) ++(0,-17) -- node[at end, above]{$y$} (y);
          \draw (z) ++(0,-17) -- node[at end, above]{$z$} (z);
          \draw (w) ++(0,-17) -- node[at end, above]{$w$} (w);

          % NOT
          \draw (notx.input) -- ++(0,3mm) -- ([yshift=3mm] notx.input -| x) node [branch]{};
          \draw (notx.output) -- (notx.output |- xend);

          \draw (noty.input) -- ++(0,3mm) -- ([yshift=3mm] noty.input -| y) node [branch]{};
          \draw (noty.output) -- (noty.output |- xend);

          \draw (notz.input) -- ++(0,3mm) -- ([yshift=3mm] notz.input -| z) node [branch]{};
          \draw (notz.output) -- (notz.output |- xend);

          \draw (notw.input) -- ++(0,3mm) -- ([yshift=3mm] notw.input -| w) node [branch]{};
          \draw (notw.output) -- (notw.output |- xend);

          % OR 1
          \draw (or1.input 1) -- (or1.input 1 -| x) node [branch]{};
          \draw (or1.input 2) -- (or1.input 2 -| y) node [branch]{};
          \draw (or1.input 3) -- (or1.input 3 -| z) node [branch]{};
          \draw (or1.input 4) -- (or1.input 4 -| notw.output) node [branch]{};

          % OR 2
          \draw (or2.input 1) -- (or2.input 1 -| x) node [branch]{};
          \draw (or2.input 2) -- (or2.input 2 -| y) node [branch]{};
          \draw (or2.input 3) -- (or2.input 3 -| notz.output) node [branch]{};
          \draw (or2.input 4) -- (or2.input 4 -| notw.output) node [branch]{};

          % OR 3
          \draw (or3.input 1) -- (or3.input 1 -| x) node [branch]{};
          \draw (or3.input 2) -- (or3.input 2 -| noty.output) node [branch]{};
          \draw (or3.input 3) -- (or3.input 3 -| z) node [branch]{};
          \draw (or3.input 4) -- (or3.input 4 -| w) node [branch]{};

          % OR 4
          \draw (or4.input 1) -- (or4.input 1 -| x) node [branch]{};
          \draw (or4.input 2) -- (or4.input 2 -| noty.output) node [branch]{};
          \draw (or4.input 3) -- (or4.input 3 -| notz.output) node [branch]{};
          \draw (or4.input 4) -- (or4.input 4 -| w) node [branch]{};

          % OR 5
          \draw (or5.input 1) -- (or5.input 1 -| notx.output) node [branch]{};
          \draw (or5.input 2) -- (or5.input 2 -| y) node [branch]{};
          \draw (or5.input 3) -- (or5.input 3 -| z) node [branch]{};
          \draw (or5.input 4) -- (or5.input 4 -| w) node [branch]{};

          % OR 6
          \draw (or6.input 1) -- (or6.input 1 -| notx.output) node [branch]{};
          \draw (or6.input 2) -- (or6.input 2 -| y) node [branch]{};
          \draw (or6.input 3) -- (or6.input 3 -| notz.output) node [branch]{};
          \draw (or6.input 4) -- (or6.input 4 -| notw.output) node [branch]{};

          % OR 7
          \draw (or7.input 1) -- (or7.input 1 -| notx.output) node [branch]{};
          \draw (or7.input 2) -- (or7.input 2 -| noty.output) node [branch]{};
          \draw (or7.input 3) -- (or7.input 3 -| z) node [branch]{};
          \draw (or7.input 4) -- (or7.input 4 -| w) node [branch]{};

          % OR 8
          \draw (or8.input 1) -- (or8.input 1 -| notx.output) node [branch]{};
          \draw (or8.input 2) -- (or8.input 2 -| noty.output) node [branch]{};
          \draw (or8.input 3) -- (or8.input 3 -| z) node [branch]{};
          \draw (or8.input 4) -- (or8.input 4 -| notw.output) node [branch]{};

          % OR 9
          \draw (or9.input 1) -- (or9.input 1 -| notx.output) node [branch]{};
          \draw (or9.input 2) -- (or9.input 2 -| noty.output) node [branch]{};
          \draw (or9.input 3) -- (or9.input 3 -| notz.output) node [branch]{};
          \draw (or9.input 4) -- (or9.input 4 -| w) node [branch]{};

          % OUTPUT
          \draw ([xshift=4mm, yshift=4mm] or1.output) node [above left] {$x + y + z + \overline{w}$} (or1.output) -- ([xshift=1.2cm] or1.output) |- (and.input 1);
          \draw ([xshift=4mm, yshift=4mm] or2.output) node [above left] {$x + y + \overline{z} + \overline{w}$} (or2.output) -- ([xshift=1cm] or2.output) |- (and.input 2);
          \draw ([xshift=4mm, yshift=4mm] or3.output) node [above left] {$x + \overline{y} + z + w$} (or3.output) -- ([xshift=0.8cm] or3.output) |- (and.input 3);
          \draw ([xshift=4mm, yshift=4mm] or4.output) node [above left] {$x + \overline{y} + \overline{z} + w$} (or4.output) -- ([xshift=0.6cm] or4.output) |- (and.input 4);
          \draw ([xshift=4mm, yshift=4mm] or5.output) node [above left] {$\overline{x} + y + z + w$} (or5.output) -- (or5.output) |- (and.input 5);
          \draw ([xshift=4mm, yshift=4mm] or6.output) node [above left] {$\overline{x} + y + \overline{z} + \overline{w}$} (or6.output) -- ([xshift=0.6cm] or6.output) |- (and.input 6);
          \draw ([xshift=4mm, yshift=4mm] or7.output) node [above left] {$\overline{x} + \overline{y} + z + w$} (or7.output) -- ([xshift=0.8cm] or7.output) |- (and.input 7);
          \draw ([xshift=4mm, yshift=4mm] or8.output) node [above left] {$\overline{x} + \overline{y} + z + \overline{w}$} (or8.output) -- ([xshift=1cm] or8.output) |- (and.input 8);
          \draw ([xshift=4mm, yshift=4mm] or9.output) node [above left] {$\overline{x} + \overline{y} + \overline{z} + w$} (or9.output) -- ([xshift=1.2cm] or9.output) |- (and.input 9);
          \draw (and.output) -- node [above] {$f$} ([xshift=0.5cm] and.output);
        \end{tikzpicture}
    }
    \caption{Funzione ottenuta dai \emph{maxterm}}
    \label{fig:funzione_ottenuta_maxterm}
\end{figure}

\subsection{Funzione semplificata}

\begin{figure}[H]
    \centering
    \scalebox{1.25}{
        \centering
        \begin{tikzpicture}
          % Variabili
          \coordinate (x) at (0,0);
          \coordinate (y) at (1.5,0);
          \coordinate (z) at (3,0);
          \coordinate (w) at (4.5,0);

          % Porte logiche
          \node [not gate US, draw, rotate=270] (notx) at ([xshift=6mm,yshift=-8mm]x){};
          \node [not gate US, draw, rotate=270] (noty) at ([xshift=6mm,yshift=-8mm]y){};
          \node [not gate US, draw, rotate=270] (notz) at ([xshift=6mm,yshift=-8mm]z){};
          \node [not gate US, draw, rotate=270] (notw) at ([xshift=6mm,yshift=-8mm]w){};

          \node [and gate US, draw, logic gate inputs=nnn] (and1) at (6.5,-2.5){$\overline{x}\overline{y}\overline{w}$};
          \node [and gate US, draw, below=of and1, yshift=3.5mm, logic gate inputs=nnn] (and2){$\overline{y}z\overline{w}$};
          \node [and gate US, draw, below=of and2, yshift=3.5mm, logic gate inputs=nnn] (and3){$\overline{x}yw$};
          \node [and gate US, draw, below=of and3, yshift=3.5mm, logic gate inputs=nnn] (and4){$yzw$};
          \node [and gate US, draw, below=of and4, yshift=3.5mm, logic gate inputs=nnnn] (and5){$x\overline{y}\overline{z}w$};
          \node [or gate US, draw, right=of and3, xshift=5mm, logic gate inputs=nnnnn] (or){};

          %  +  +  +  +

          % INPUT
          \draw (x) ++(0,-10) node[](xend){} -- node[at end, above]{$x$} (x);
          \draw (y) ++(0,-10) -- node[at end, above]{$y$} (y);
          \draw (z) ++(0,-10) -- node[at end, above]{$z$} (z);
          \draw (w) ++(0,-10) -- node[at end, above]{$w$} (w);

          % NOT
          \draw (notx.input) -- ++(0,3mm) -- ([yshift=3mm] notx.input -| x) node [branch]{};
          \draw (notx.output) -- (notx.output |- xend);

          \draw (noty.input) -- ++(0,3mm) -- ([yshift=3mm] noty.input -| y) node [branch]{};
          \draw (noty.output) -- (noty.output |- xend);

          \draw (notz.input) -- ++(0,3mm) -- ([yshift=3mm] notz.input -| z) node [branch]{};
          \draw (notz.output) -- (notz.output |- xend);

          \draw (notw.input) -- ++(0,3mm) -- ([yshift=3mm] notw.input -| w) node [branch]{};
          \draw (notw.output) -- (notw.output |- xend);

          % AND 1
          \draw (and1.input 1) -- (and1.input 1 -| notx.output) node [branch]{};
          \draw (and1.input 2) -- (and1.input 2 -| noty.output) node [branch]{};
          \draw (and1.input 3) -- (and1.input 3 -| notw.output) node [branch]{};

          % AND 2
          \draw (and2.input 1) -- (and2.input 1 -| noty.output) node [branch]{};
          \draw (and2.input 2) -- (and2.input 2 -| z) node [branch]{};
          \draw (and2.input 3) -- (and2.input 3 -| notw.output) node [branch]{};

          % AND 3
          \draw (and3.input 1) -- (and3.input 1 -| notx.output) node [branch]{};
          \draw (and3.input 2) -- (and3.input 2 -| y) node [branch]{};
          \draw (and3.input 3) -- (and3.input 3 -| w) node [branch]{};

          % AND 4
          \draw (and4.input 1) -- (and4.input 1 -| y) node [branch]{};
          \draw (and4.input 2) -- (and4.input 2 -| z) node [branch]{};
          \draw (and4.input 3) -- (and4.input 3 -| w) node [branch]{};

          % AND 5
          \draw (and5.input 1) -- (and5.input 1 -| x) node [branch]{};
          \draw (and5.input 2) -- (and5.input 2 -| noty.output) node [branch]{};
          \draw (and5.input 3) -- (and5.input 3 -| notz.output) node [branch]{};
          \draw (and5.input 4) -- (and5.input 4 -| w) node [branch]{};

          % OUTPUT
          \draw (and1.output) -- ([xshift=0.5cm] and1.output) |- (or.input 1);
          \draw (and2.output) -- ([xshift=0.3cm] and2.output) |- (or.input 2);
          \draw (and3.output) -- (and3.output) |- (or.input 3);
          \draw (and4.output) -- ([xshift=0.3cm] and4.output) |- (or.input 4);
          \draw (and5.output) -- ([xshift=0.5cm] and5.output) |- (or.input 5);
          \draw (or.output) -- node [above] {$f$} ([xshift=0.5cm] or.output);
        \end{tikzpicture}
    }
    \caption{Funzione semplificata}
    \label{fig:funzione_semplificata}
\end{figure}

\newpage
\section{Dichiarazione finale}
Il lavoro di cui sopra è stato svolto da me in completa autonomia.

\end{document}
